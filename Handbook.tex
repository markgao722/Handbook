\documentclass[10pt,a4paper,fleqn]{article} % PREAMBLE ----------------------------------
% Comma separated list of parameters
% One choice of class containing presets; most of the time article will suffice

% Package installation ---

% Table of contents 
\usepackage{tocloft} % table of content dots
\renewcommand{\cftsecleader}{\cftdotfill{\cftdotsep}} % table of content dots
\usepackage[hidelinks]{hyperref} % clickable contents

% General formatting
\usepackage{blindtext} % generates lorem ipsum text
\usepackage{color} % colored font
\usepackage{xcolor}
\usepackage[top=0.75in, left=1in, right=1in, bottom=1in]{geometry} % adjust margins precisely
\usepackage{enumerate} % needed for roman numeral lists. Other types include itemize, description
\usepackage{soul} % the default underline doesn't allow line breaks; use \ul{}
\usepackage[english]{babel} % all three lines below used to fixed quotes facing wrong way " " 
\usepackage[autostyle, english=american]{csquotes}
\MakeOuterQuote{"}

% Code
\usepackage{listings}
\lstdefinestyle{javac}{ % settings for style=javac
  breaklines=true,
  frame=single,
  language=Java, % custom defined colors for the Java option
  showstringspaces=false,
  basicstyle=\footnotesize\ttfamily,
  keywordstyle=\bfseries\color{orange!90!black},
  commentstyle=\itshape\color{black!25},
  stringstyle=\color{green!40!black},
}

% Math
\usepackage{amsmath} % needed for aligning multi-line equations
\usepackage{amssymb} % needed for R real numbers, etc.
\makeatletter % fix to allow long divider bars in the matrix environment (for matrix equations)
\renewcommand*\env@matrix[1][*\c@MaxMatrixCols c]{%
  \hskip -\arraycolsep
  \let\@ifnextchar\new@ifnextchar
  \array{#1}}
\makeatother

% Images
\usepackage{float} % needed for aligning images
\usepackage{graphicx} % needed for inserting images
\graphicspath{{./imgs}} % path to images folder relative to this .tex file

% State diagrams
\usepackage{tikz}
\usetikzlibrary{snakes, automata, positioning, shapes}

% Attention boxes
\usepackage{tcolorbox}

\begin{document} % DOCUMENT BODY --------------------------------------------------------

\title{Handbook}
\author{Mark Gao}
\date{2023}
\maketitle

\newpage

\tableofcontents

\newpage

% Part - math or CS
% Section - individual course
% Subsection - course module
% Subsubsection - theorem

\part{Math}

\section{Linear Algebra} % ==============================================================

\subsection{Matrix Operations} % -------------------------------------------------------
The \textbf{addition} and subtraction of matrices happens element-wise. Take two $2\times2$ matrices $A$ and $B$:
\begin{equation}
    A + B = 
    \begin{pmatrix}
        a_{11} + b_{11} & a_{12} + b_{12} \\
        a_{21} + b_{21} & a_{22} + b_{22} \\
    \end{pmatrix}
\end{equation}
The multiplication operation is known as the \textbf{dot product}.
\begin{equation}
    A \cdot B = 
    \begin{pmatrix}
        a_{11} b_{11} + a_{12} b_{21} & a_{11} b_{12} + a_{12} b_{22} \\
        a_{21} b_{11} + a_{22} b_{21} & a_{21} b_{12} + a_{22} b_{22} \\
    \end{pmatrix}
\end{equation}
The division operation does not exist for matrices. There is no inverse operation to
the dot product.
Note that matrix multiplication is \textit{not commutative}. Multiplication is carried out right to left.
\begin{equation*}
    ABC = A(BC)
\end{equation*}
The equivalent concept to "dividing away" on two sides of an equation is to right-multiply an inverse.
\begin{equation*}
    B = ADC
\end{equation*}
The \textbf{transpose} of a matrix is flipped along its diagonal.
Where the matrix is not square, this diagonal is imagined.
For example given the $2\times3$ matrix $C$:
\begin{equation}
    \begin{pmatrix}
        c_{11} & c_{12} & c_{13} \\
        c_{21} & c_{22} & c_{23} \\
    \end{pmatrix}^T
    =
    \begin{pmatrix}
        c_{11} & c_{21} \\
        c_{12} & c_{22} \\
        c_{13} & c_{23} \\
    \end{pmatrix}
\end{equation}

\subsection{Inverse} % ------------------------------------------------------------------
The \textbf{inverse} of a matrix should satisfy $AA^{-1}=I$. The inverse is also the reciprocal of its determinant.
\begin{equation}
    A^{-1} = \frac{1}{|A|}
\end{equation}
For any $n\times n$ matrix, the inverse can also be found via Gauss-Jordan elimination.
When the LHS is reduced to $I$, the RHS is the inverse of $A$.
\begin{equation}
    \begin{pmatrix}[ccc|ccc]
        a_{11} & a_{12} & a_{13} & 1 & 0 & 0 \\
        a_{21} & a_{22} & a_{23} & 0 & 1 & 0 \\
        a_{31} & a_{32} & a_{33} & 0 & 0 & 1 \\
    \end{pmatrix}
\end{equation}

\subsection{Determinant} % --------------------------------------------------------------


\subsection{Cross Proudct} % ------------------------------------------------------------
The \textbf{cross product} is a new operation to matrices. For a $2\times2$ matrix,
the cross product is the same as its determinant.
\begin{equation}
    A \times B
    =
    \det{A} - \det{B}
    =
    a_{11} a_{22} - a_{12} a_{21} - \left( b_{} b_{} - b_{} b_{} \right)
\end{equation}
The cross product is more meaningful for matrices $3\times3$ or bigger.
\begin{equation}
    A \times B
    =
    i \begin{pmatrix}
        
    \end{pmatrix}
    +
    j \begin{pmatrix}
        
    \end{pmatrix}
    +
    k \begin{pmatrix}
        
    \end{pmatrix}
\end{equation}

\subsection{Square Matrices} % ----------------------------------------------------------
A \textbf{diagonal matrix} is a square matrix which is comprised of non-zero values
along the diagonal and zeros everywhere else.
The \textbf{identity matrix} is a square matrix with $1$'s on the diagonal.
\begin{equation}
    I = \begin{pmatrix}
        1 & 0 & \cdots & 0 \\
        0 & 1 & \cdots & 0 \\
        \vdots & \vdots & \ddots & 0 \\
        0 & 0 & 0 & 1 \\
    \end{pmatrix}    
\end{equation}

\subsection{Eigenvalues and Eigenvectors} % ---------------------------------------------


\section{Calculus} % ====================================================================

\subsection{Derivative and integral foundations} % --------------------------------------

\subsection{Multivariable Topics} % -----------------------------------------------------


\section{Probability} % =================================================================

\subsection{Probabilitly foundations} % -------------------------------------------------

\subsection{Markov Chains} % ------------------------------------------------------------

\subsection{Renewal Processes} % --------------------------------------------------------

\newpage

\part{Computer Science}

\section{Object Oriented Programming} % =================================================
This section will mostly be covered in Java, with analogies drawn to Python. 
This section will cover:
\begin{itemize}
    \item Scope and encapsulation
    \item Polymorphism
\end{itemize}

\subsection{Java foundations} % ---------------------------------------------------------

\begin{lstlisting}[language=Java,style=javac]
    class MyClass{
        private int x = 123;

        public MyClass(int a) {
            ...
        }
    }
\end{lstlisting}
wtf?
hello


\subsection{Scope \& Encapsulation} % ---------------------------------------------------


\section{Data Structures \& Algorithms} % ===============================================

\section{Machine Learning} % ============================================================

\newpage

\part {Finance}

\section{Fixed Income} % ================================================================

\end{document}