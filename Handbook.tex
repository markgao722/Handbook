\documentclass[10pt,a4paper,fleqn]{article} % PREAMBLE ----------------------------------
% Comma separated list of parameters
% One choice of class containing presets; most of the time article will suffice

% Package installation ---

% Table of contents 
\usepackage{tocloft} % table of content dots
\renewcommand{\cftsecleader}{\cftdotfill{\cftdotsep}} % table of content dots
\usepackage[hidelinks]{hyperref} % clickable contents

% General formatting
\usepackage{blindtext} % generates lorem ipsum text
\usepackage{color} % colored font
\usepackage{xcolor}
\usepackage[top=0.75in, left=1in, right=1in, bottom=1in]{geometry} % adjust margins precisely
\usepackage{enumerate} % needed for roman numeral lists. Other types include itemize, description
\usepackage{soul} % the default underline doesn't allow line breaks; use \ul{}
\usepackage[english]{babel} % all three lines below used to fixed quotes facing wrong way " " 
\usepackage[autostyle, english=american]{csquotes}
\MakeOuterQuote{"}

% Code
\usepackage{listings}
\lstdefinestyle{javac}{ % settings for style=javac
  breaklines=true,
  frame=single,
  language=Java, % custom defined colors for the Java option
  showstringspaces=false,
  basicstyle=\footnotesize\ttfamily,
  keywordstyle=\bfseries\color{orange!90!black},
  commentstyle=\itshape\color{black!25},
  stringstyle=\color{green!40!black},
}

% Math
\usepackage{amsmath} % needed for aligning multi-line equations
\usepackage{amssymb} % needed for R real numbers, etc.

% Images
\usepackage{float} % needed for aligning images
\usepackage{graphicx} % needed for inserting images
\graphicspath{{./imgs}} % path to images folder relative to this .tex file

% State diagrams
\usepackage{tikz}
\usetikzlibrary{snakes, automata, positioning, shapes}

% Attention boxes
\usepackage{tcolorbox}

\begin{document} % DOCUMENT BODY --------------------------------------------------------

\title{Handbook}
\author{Mark Gao}
\date{2023}
\maketitle

\newpage

\tableofcontents

\newpage

% Part - math or CS
% Section - individual course
% Subsection - course module
% Subsubsection - theorem

\part{Math}

\section{Linear Algebra} % ==============================================================

\subsection{Matrix foundations} % -------------------------------------------------------
This section will cover:
\begin{itemize}
    \item Matrix operations (addition/subtraction, multiplication, transpose, inverse)
    \item Gauss-Jordan elimination
    \item Notable matrices (identity, diagonal)
\end{itemize}

\subsubsection{Basic Matrix Operations}

\subsection{Eigenvalues and eigenvectors} % ---------------------------------------------


\section{Calculus} % ====================================================================

\subsection{Derivative and integral foundations} % --------------------------------------

\subsection{Multivariable Topics} % -----------------------------------------------------


\section{Probability} % =================================================================

\subsection{Probabilitly foundations} % -------------------------------------------------

\subsection{Markov Chains} % ------------------------------------------------------------

\subsection{Renewal Processes} % --------------------------------------------------------

\newpage

\part{Computer Science}

\section{Object Oriented Programming} % =================================================
This section will mostly be covered in Java, with analogies drawn to Python. 
This section will cover:
\begin{itemize}
    \item Scope and encapsulation
    \item Polymorphism
\end{itemize}

\subsection{Java foundations} % ---------------------------------------------------------

\begin{lstlisting}[language=Java,style=javac]
    class MyClass{
        private int x = 123;

        public MyClass(int a) {
            ...
        }
    }
\end{lstlisting}
wtf?
hello


\subsection{Scope \& Encapsulation} % ---------------------------------------------------


\section{Data Structures \& Algorithms} % ===============================================

\section{Machine Learning} % ============================================================

\newpage

\part {Finance}

\section{Fixed Income} % ================================================================

\end{document}